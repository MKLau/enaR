%%%%%%%%%%%%%%%%%%%%%%%%%%%%%%%%%%%%%%%%%%%
%
%  CFRE ENA manuscript
%  for Ecological Modeling (sp. ed.)
%  Dave Hines 5/15/2013
%
%%%%%%%%%%%%%%%%%%%%%%%%%%%%%%%%%%%%%%%%%%%

\documentclass[]{article}
%\usepackage[super, sort]{natbib}
%\bibpunct{(}{)}{;}{a}{,}{,} % required for natbib
\bibliographystyle{elsart-harv}
\usepackage{amsmath,amssymb,amsthm,amsfonts}
\usepackage[]{graphicx}
\usepackage{lineno}
\usepackage{natbib} %bib package
\usepackage{amsmath,amssymb} %math package 
\usepackage{setspace}  %necessary for double spacing
%\doublespacing
\usepackage{hyperref}

\begin{document}

\title{\texttt{enaR} problem - enaEnviron()}
\author{Hines}
\date{October 1, 2013}
\maketitle

\section{Problem}

Here, I document an issue with the enaEnviron() function for
\texttt{enaR v2.0} so that we can deal with it in the future.  The
function returns outputs that make little since (sometimes the output
even has negative flows, as I will demonstrate) and do not represent
the correct partitioning of the flows in a network.  The issue arose
somewhere between \texttt{enaR} versions 1.01 and 2.0, and I think it
may be related to the addition of the 'rc' and 'school' orientations,
but I can't confirm this.  Below I provide examples of how the
software was working in version 1.01, but is failing in version 2.0 in
both the oligohaline Cape Fear River Estuary (CFRE) nitrogen cycling
model and the oyster model.

\subsection{Oyster reef model}

Here I show the results of the unit environ calculations for the oyster
reef model included in the \texttt{enaR} package using different
versions of \texttt{R} and \texttt{enaR}.   I show the input microbiota environ
for the oyster reef model as an example for each software configuration.

\subsubsection{\texttt{R} version 2.15.3, \texttt{enaR} version 1.01}

Oyster reef model, unit microbiota input environ:

\[
\begin{array}{ccccccc}
-1 & 0 & 0 & 0 & 0 & 0 & 1 \\
0 & -1.1 & 0 & 0 & 0 & 1.1 & 0 \\
0 & 0.04 & -.29 & 0 & 0 & 0.25 & 0 \\
0 & 0.05 & 0.03 & -0.12 & 0 & 0.03 & 0 \\
0.02 & 0 & 0 & 0.01 & -0.02 & 0 & 0 \\
0.98 & 0 & 0.26 & 0.11 & 0.02 & -1.4 & 0 \\
0 & 1 & 0 & 0 & 0 & 0 & 0 \\
\end{array}
\]

The order of the compartments here is 1) filter feeders, 2)
microbiota, 3) meiofauna, 4) deposit feeders, 5) predators, 6)
Deposited detritus.  The seventh column is the input vector (z) and
the seventh row is the output vector (y).  This is the correct version
of the input environ for microbiota.  Notice that there is only one
output (microbiota).  There is only one input (filter feeders) because
that is the only input to the system.  The negative values along the
principle diagonal are the negative throughflows for each compartment.

\subsubsection{\texttt{R} version 3.0.1, \texttt{enaR} version 2.0}

Oyster reef model, unit microbiota input environ:  

\[
\begin{array}{ccccccc}
0.055 & 0 & 0 & 0 & -0.0004 & -0.15 & -0.039 \\
0 & -.63 & 0.32 & 0.32 & 0 & 0 & 0 \\
0 & 0 & 0.03 & -0.004 & 0 & -0.02 & 0 \\
0 & 0 & 0 & -0.102 & 0.008 & 0.09 & 0 \\
0 & 0 & 0 & 0 & 0.189 & -0.189 & 0 \\
0 & -0.17 & -0.15 & -0.01 & 0 & 0.33 & 0 \\
-0.05 & 0.8 & -0.19 & -0.19 & -0.19 & -0.19 & 0 \\
\end{array}
\]

For the above environ, \texttt{set.orient} was set to
\texttt{``school''}.  Clearly the environ produced by the above
calculation is FUBAR, as there are flows where there should be none
and there are some flows that are negative.  This problem persists
with \texttt{set.orient} set to \texttt{``rc''}, and the results are
identical, just in the opposite orientation.

\subsection{Oligohaline CFRE nitrogen cycling model}

Here I show the results of the realized environ calculations for the
denitrification environ in the CFRE nitrogen cycling model.  All of
the environs are messed up, but here I show the denitrification
environ because it may help to diagnose the problem.  This model has
14 nodes, 6 of which are ``ghost nodes'' that were added to the original
model structure to maintain resolution of multiple removal processes
from a single node.  The denitrification environ uses on such ghost
node, and gives a very odd result.

\subsubsection{\texttt{R} version 2.15.3, \texttt{enaR} version 1.01}

Oligohaline CFRE nitrogen cycling model, realized denitrification input environ:  

\tiny

\[
\begin{array}{ccccccccccccccc}
  -0.03 & 0 & 0.004 & 0.006 & 0.17 & 0 & 0 & 0 & 0 & 0 & 0 & 0 & 0 & 0 &
  0.16  \\
  0.02 & -7.75 & 0 & 0 & 0 & 0.16 & 0 & 0 & 0 & 0 & 0 & 0 & 0 & 0 & 7.7
  \\
  0.04 & 0.22 & -3.15 & 0.17 & 0 & 0 & 2.72 & 0 & 0 & 0 & 0 & 0 & 0 & 0
  & 0 \\
  0 & 0 & 0.08 & -10.4 & 0 & 0 & 0 & 4.39 & 0 & 0 & 0 & 0 & 0 & 0 & 5.99
  \\
  0.28 & 0 & 0 & 0 & -82.6 & 2.03 & 7.67 & 7.83 & 0 & 0 & 0 & 0 & 0 & 0
  & 64.7 \\
  0 & 7.53 & 0 & 0 & 76.9 & -176.86 & 0 & 0 & 0 & 0 & 0 & 0 & 0 & 0 &
  92.4 \\
  0 & 0 & 3.07 & 0 & 5.49 & 2.81 & -13.49 & 2.11 & 0 & 0 & 0 & 0 & 0 & 0
  & 0 \\
  0 & 0 & 0 & 10.29 & 0 & 0 & 3.09 & -14.34 & 0 & 0 & 0 & 0 & 0 & 0 &
  0.95 \\
  0 & 0 & 0 & 0 & 0 & 0 & 0 & 0 & 0 & 0 & 0 & 0 & 0 & 0 & 0 \\
  0 & 0 & 0 & 0 & 0 & 0 & 0 & 0 & 0 & 0 & 0 & 0 & 0 & 0 & 0 \\
  0 & 0 & 0 & 0 & 0 & 172 & 0 & 0 & 0 & 0 & -172 & 0 & 0 & 0 & 0 \\
  0 & 0 & 0 & 0 & 0 & 0 & 0 & 0 & 0 & 0 & 0 & 0 & 0 & 0 & 0 \\
  0 & 0 & 0 & 0 & 0 & 0 & 0 & 0 & 0 & 0 & 0 & 0 & 0 & 0 & 0 \\
  0 & 0 & 0 & 0 & 0 & 0 & 0 & 0 & 0 & 0 & 0 & 0 & 0 & 0 & 0 \\
  0 & 0 & 0 & 0 & 0 & 0 & 0 & 0 & 0 & 0 & 172 & 0 & 0 & 0 & 0 \\
\end{array}
\]

\normalsize

The nodes in the above environ calculation are 1) W-NH$_4$, 2)
W-NO$_x$, 3) W-M, 4) W-ON, 5) S-NH$_4$, 6) S-NO$_x$, 7) S-M, 8) S-ON,
9) anammox-ghost1, 10) anammox-ghost2, 11) denitrification-ghost, 12)
S-NO$_x$-buirial-ghost, 13) S-M-burial-ghost, and 14) S-ON-burial-ghost.
This is the proper realized input environ for denitrification in the
CFRE model.  There is only one output (denitrification), and all
the proper inputs and internal fluxes show up.

\subsubsection{\texttt{R} version 3.0.1, \texttt{enaR} version 2.0}

Oligohaline CFRE nitrogen cycling model, realized denitrification
input environ:  

\tiny

\[
\begin{array}{ccccccccccccccc}
  0 & 0 & 0 & 0 & 0 & 0 & 0 & 0 & 0 & 0 & 0 & 0 & 0 & 0 & 0 \\
  0 & 0 & 0 & 0 & 0 & 0 & 0 & 0 & 0 & 0 & 0 & 0 & 0 & 0 & 0 \\
  0 & 0 & 0 & 0 & 0 & 0 & 0 & 0 & 0 & 0 & 0 & 0 & 0 & 0 & 0 \\
  0 & 0 & 0 & 0 & 0 & 0 & 0 & 0 & 0 & 0 & 0 & 0 & 0 & 0 & 0 \\
  0 & 0 & 0 & 0 & 0 & 0 & 0 & 0 & 0 & 0 & 0 & 0 & 0 & 0 & 0 \\
  0 & 0 & 0 & 0 & 0 & 0 & 0 & 0 & 0 & 0 & 0 & 0 & 0 & 0 & 0 \\
  0 & 0 & 0 & 0 & 0 & 0 & 0 & 0 & 0 & 0 & 0 & 0 & 0 & 0 & 0 \\
  0 & 0 & 0 & 0 & 0 & 0 & 0 & 0 & 0 & 0 & 0 & 0 & 0 & 0 & 0 \\
  0 & 0 & 0 & 0 & 0 & 0 & 0 & 0 & 0 & 0 & 0 & 0 & 0 & 0 & 0 \\
  0 & 0 & 0 & 0 & 0 & 0 & 0 & 0 & 0 & 0 & 0 & 0 & 0 & 0 & 0 \\
  0 & 0 & 0 & 0 & 0 & 0 & 0 & 0 & 0 & 0 & -172 & 0 & 0 & 0 & 172 \\
  0 & 0 & 0 & 0 & 0 & 0 & 0 & 0 & 0 & 0 & 0 & 0 & 0 & 0 & 0 \\
  0 & 0 & 0 & 0 & 0 & 0 & 0 & 0 & 0 & 0 & 0 & 0 & 0 & 0 & 0 \\
  0 & 0 & 0 & 0 & 0 & 0 & 0 & 0 & 0 & 0 & 0 & 0 & 0 & 0 & 0 \\
  0 & 0 & 0 & 0 & 0 & 0 & 0 & 0 & 0 & 0 & 172 & 0 & 0 & 0 & 0 \\
\end{array}
\]

\normalsize

In the above environ, there is no system activity, accept for in the
ghost node itself.  This may help to identify the problem, as somehow
we are adding an input to this node and cutting it off from the rest
of the network.  Structurally, there is no boundary input to node 11
(the denitrification ghost node), but here it shows that 172 units of
flow enter from the boundary, then immediately exit the node.  Are we
assigning the boundary inputs wrong somewhere?

\section{Working code: \texttt{CFREcoupling}}

Below is the code I used for the \texttt{CFREcoupling} function I used
to conduct the sensitivity analysis for the L\&O paper.  This function
calculated the desired environs by hand, but did so based off of the
\texttt{enaEnviron} function.  For whatever reason, the code works in
the \texttt{CFREcoupling} function, but not in the \texttt{enaEnviron}
function.

code:

\begin{verbatim}
# Coupling Calculation
# INPUT = 14 node CFRE network object
# OUTPUT = percent coupling of: NTR-DNT | NTR-AMX | DNRA-AMX
# 
# DEH August 2013
# ---------------------------------------------------

CFREcoupling <- function(x = 'network object', err.tol = 1e-10,balance.override=FALSE){
                                        #check for network class
  if (class(x) != 'network'){warning('x is not a network class object')}

                                          #Check for balancing
  if (balance.override){}else{
    if (any(list.network.attributes(x) == 'balanced') == FALSE){x%n%'balanced' <- ssCheck(x)}
    if (x%n%'balanced' == FALSE){warning('Model is not balanced'); stop}
  }

  F <- enaFlow(x)   # calculate enaFlow
                                    ##### Denitrification environ #####
    
      dNP <- diag(F$NP[11,]) #diagonalized N matrix
      DNT <- dNP %*% F$GP         # calculate internal environ flows
      DNT <- DNT - dNP      # place negative environ throughflows on the principle diagonal
      DNT <- cbind(DNT,apply((-DNT),1,sum)) #attach z column
      DNT <- rbind(DNT,c(apply((-DNT),2,sum))) #attach y row
      DNT[15,15] <- 0 #add zero to bottom right corner to complete matrix
      DNT[abs(DNT) < err.tol] <- 0 #ignor numerical error
          #add labels to matrices
          labels <- c(rownames(F$GP))
          colnames(DNT) = c(labels, 'z')
          rownames(DNT) = c(labels, 'y')
	 DNT = DNT * unpack(x)$y[11]
    
         ##### Anammox environ #####
    
      dNP <- diag(F$NP[9,]) #diagonalized N matrix
      AMX5 <- dNP %*% F$GP         # calculate internal environ flows
      AMX5 <- AMX5 - dNP      # place negative environ throughflows on the principle diagonal
      AMX5 <- cbind(AMX5,apply((-AMX5),1,sum)) #attach z column
      AMX5 <- rbind(AMX5,c(apply((-AMX5),2,sum))) #attach y row
      AMX5[15,15] <- 0 #add zero to bottom right corner to complete matrix
      AMX5[abs(AMX5) < err.tol] <- 0 #ignor numerical error
          #add labels to matrices
          labels <- c(rownames(F$GP))
          colnames(AMX5) = c(labels, 'z')
          rownames(AMX5) = c(labels, 'y')
	 AMX5 = AMX5 * unpack(x)$y[9]
    
      dNP <- diag(F$NP[10,]) #diagonalized N matrix
      AMX6 <- dNP %*% F$GP         # calculate internal environ flows
      AMX6 <- AMX6 - dNP      # place negative environ throughflows on the principle diagonal
      AMX6 <- cbind(AMX6,apply((-AMX6),1,sum)) #attach z column
      AMX6 <- rbind(AMX6,c(apply((-AMX6),2,sum))) #attach y row
      AMX6[15,15] <- 0 #add zero to bottom right corner to complete matrix
      AMX6[abs(AMX6) < err.tol] <- 0 #ignor numerical error
          #add labels to matrices
          labels <- c(rownames(F$GP))
          colnames(AMX6) = c(labels, 'z')
          rownames(AMX6) = c(labels, 'y')
	 AMX6 = AMX6 * unpack(x)$y[10]
	 
	 AMX = AMX5 + AMX6
	    
     # Nitrification coupled to denitrification
     NTR.DNT = (DNT[6,5]/(DNT[15,11]+DNT[7,6]+DNT[2,6]+DNT[5,6]))
     #NTR.DNT.A = NTR.DNT * unpack(x)$y[11]

     # Nitrification coupled to anammox
     NTR.AMX = (AMX[6,5]/((AMX[15,9]+AMX[15,10])+AMX[7,6]+AMX[2,6]+AMX[5,6]))
     #NTR.AMX.A = NTR.AMX * (unpack(x)$y[9] + unpack(x)$y[10])

     # DNRA coupled to anammox
     DNRA.AMX = (AMX[5,6]/((AMX[15,9]+AMX[15,10])+AMX[7,6]+AMX[2,6]+AMX[5,6]))
     #DNRA.AMX.A = DNRA.AMX * (unpack(x)$y[9] + unpack(x)$y[10])


     # Set up output
     out = c(NTR.DNT, NTR.AMX, DNRA.AMX)

  return(out)

}
\end{verbatim}


\section{Working code: \texttt{environ}}

Below is the code for the \texttt{environ} function in \texttt{enaR}
version 1.01.  

code:


\begin{verbatim}
function (x = "network object", input = TRUE, output = TRUE, 
    err.tol = 1e-10, balance.override = FALSE) 
{
    if (class(x) != "network") {
        warning("x is not a network class object")
    }
    if (balance.override) {
    }
    else {
        if (any(list.network.attributes(x) == "balanced") == 
            FALSE) {
            x %n% "balanced" <- ssCheck(x)
        }
        if (x %n% "balanced" == FALSE) {
            warning("Model is not balanced")
            stop
        }
    }
    F <- enaFlow(x)
    if (input == TRUE) {
        EP <- list()
        for (i in (1:nrow(F$NP))) {
            dNP <- diag(F$NP[i, ])
            EP[[i]] <- dNP %*% F$GP
            EP[[i]] <- EP[[i]] - dNP
            EP[[i]] <- cbind(EP[[i]], apply((-EP[[i]]), 1, sum))
            EP[[i]] <- rbind(EP[[i]], c(apply((-EP[[i]]), 2, 
                sum)))
            EP[[i]][nrow(EP[[i]]), ncol(EP[[i]])] <- 0
            EP[[i]][abs(EP[[i]]) < err.tol] <- 0
        }
        names(EP) <- rownames(F$GP)
    }
    if (output == TRUE) {
        E <- list()
        for (i in (1:nrow(F$N))) {
            dN <- diag(F$N[, i])
            E[[i]] <- F$G %*% dN
            E[[i]] <- E[[i]] - dN
            E[[i]] <- cbind(E[[i]], apply((-E[[i]]), 1, sum))
            E[[i]] <- rbind(E[[i]], c(apply((-E[[i]]), 2, sum)))
            E[[i]][nrow(E[[i]]), ncol(E[[i]])] <- 0
            E[[i]][abs(E[[i]]) < err.tol] <- 0
        }
        names(E) <- rownames(F$G)
    }
    if (input == TRUE & output == TRUE) {
        out <- list(input = EP, output = E)
    }
    else if (input == TRUE & output == FALSE) {
        out <- EP
    }
    else if (input == FALSE & output == TRUE) {
        out <- E
    }
    return(out)
}
\end{verbatim}

\end{document}